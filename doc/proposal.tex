\documentclass[11pt,oneside,a4paper]{article}

\usepackage[utf8]{inputenc}
\usepackage[swedish]{babel}
\usepackage{enumerate}
\usepackage{hyperref}

\begin{document}
\title{Avancerad individuell kurs, 15 hp}
\author{Erik Helin \& Adam Renberg}
\date{\today}
\maketitle

\section{Ämne}
Kursen kommer vara en projektkurs inom operativsystem (OS).

\section{Handledare}
Torbjörn Granlund

\section{Syfte}
Syftet med kursen är att bygga vidare på kursen ID2206 Operativsystem, 
7.5 hp, med en mer praktisk inriktning. Kursen kommer att fokusera 
på implementationen av en OS kärna för att tillämpa den teori som lärs 
ut i kursen ID2206.

\section{Förkunskapskrav}
Kursen ID2206 Operativsystem samt IS1200 Datorteknik

\section{Tid}
Kursen går i period 3 och det kommer vara den enda kurs som vi läser under
den perioden. Vi räknar med att lägga 40 timmar i veckan under hela period 3.

\section{Kursmål}
Kursen mål är:
\begin{itemize}
    \item Ge kunskap inom x86 assembly programmering
    \item Ge förståelse för hur drivrutiner implementeras
    \item Ge förståelse för hur timers implementeras
    \item Ge god insikt i svårigheterna med att implementera en OS kärna
    \item Ge god förståelse för hur virtuellt minne implementeras
    \item Ge god förståelse för hur ett fleranvändar-OS implementeras
    \item Ge väldigt god förståelse för programmering i C
\end{itemize}

\section{Examination}
För att kursen ska anses avklarad krävs det att OS kärnan kan starta i
simulatorn. När OS väl har startat skall text kunna skrivas ut på
skärmen och användaren ska kunna använda tangentbordet. För endast godkänt
krävs alltså \emph{inte} stöd för userspace, processer eller virtuellt minne.
Vidare, för att betona vikten av OS kärnor måste vara korrekta måste koden vara
väl dokumenterad och tydlig för att uppnå godkänt betyg.
Slutligen skall även rapport lämnas in som sammanfattar projektet.

För högsta betyg krävs att det finns en distinktion mellan kernelspace och
userspace, att flera program kan exekveras samtidigt samt att virtuellt minne är
implementerat. Notera att det \emph{ej} krävs ett filsystem för att nå högsta
betyg, men för att kunna demonstrera att flera program kan köras samtidigt
kommer förmodligen ett readonly RAM filsystem att behövas.

Betyg mellan godkänt och högsta betyg delas ut beroende på hur mycket som har
hunnit bli implementerat av kriterierna för högsta betyg.

\section{Förenklingar}
För att det skall vara rimligt att bli klar med projektet under period 3 så
kommer en del förenklingar att göras.
Projektet behöver ej köra på riktig hårdvara, det räcker med om OS:t går att
köra i x86 emulatorn bochs~\cite{bochs}. Projektet kommer även att använda
GRUB~\cite{grub} som bootloader. Projektet kommer \emph{förmodligen} använda en
simplifierad segmenteringsmodell. Vi kommer bara att använda minsta möjliga
antal segment, 4 stycken, för att underlätta segmenthanteringen.

\section{Kursplan}
\begin{enumerate}
    \item Sätta upp en utvecklingsmiljö (lyckas bygga minsta möjliga kärna och
    starta den i bochs)
    \item Implementera drivrutiner för konsol, timer och tangentbord.
    \item Implementera readonly RAM filsystem (eller integrera befintligt)
    \item Implementera virtuellt minne
    \item Implementera stöd för flera användare
\end{enumerate}

\section{Kurslitteratur}
Finns till stora delar tillgänglig på internet, wikin OSDev~\cite{os}
 har väldigt mycket relevant information.
Det finns en liknande kurs på CMU~\cite{cmu} som också
har mycket bra information.

\begin{thebibliography}{9}
    \bibitem{bochs}
    \url{http://bochs.sourceforge.net/}

    \bibitem{grub}
    \url{http://www.gnu.org/software/grub/}

    \bibitem{cmu}
    \url{http://www.cs.cmu.edu/~410/}

    \bibitem{os}
    \url{http://wiki.osdev.org/Main_Page}
\end{thebibliography}

\end{document}
